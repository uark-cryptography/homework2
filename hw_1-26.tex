\documentclass[12pt]{article}
\usepackage[latin1]{inputenc}

\renewcommand{\qedsymbol}{$\blacksquare$}

\title{1.26}
\author{Spencer Vaughn}


\begin{document}
1.26

\{p$_{\text{1}}$, p$_{\text{2}}$, ..., p$_{\text{r}}\}$ is a set of prime numbers.

N = \{p$_{\text{1}}$, p$_{\text{2}}$, ..., p$_{\text{r}}\}$ + 1. \newline

Thm.: N is divisble by some prime not in the set \{p$_{\text{1}}$, p$_{\text{2}}$, ..., p$_{\text{r}}\}$ .

Proof: Let M be prime. M divides N. Suppose M is in the set of prime 

numbers, then M divides N and M divides both the set of prime numbers 

and 1, which is a contradiction, since M can not divide both a prime 

number and 1. Therefore, M is not in the set of prime numbers, 

\{p$_{\text{1}}$, p$_{\text{2}}$, ..., p$_{\text{r}}\}$. QED \newline 

Thm.: There are infinitely many prime numbers.

Proof: Suppose the number of primes is finite, then M must be in the set 

of prime numbers \{p$_{\text{1}}$, p$_{\text{2}}$, ..., p$_{\text{r}}\}$. From the previous proof, M was found 

to not be in the set of prime numbers \{p$_{\text{1}}$, p$_{\text{2}}$, ..., p$_{\text{r}}\}$, then by contradiction

the number of primes is infinite. QED


\end{document}


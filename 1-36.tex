\documentclass[12pt]{article}
\usepackage{amssymb}
\usepackage{amsmath}
\usepackage{graphicx}
\usepackage{hyperref}
\usepackage[latin1]{inputenc}

\newcommand{\divides}{\bigm|}
\newcommand{\ndivides}{%
  \mathrel{\mkern.5mu % small adjustment
    % superimpose \nmid to \big|
    \ooalign{\hidewidth$\big|$\hidewidth\cr$\nmid$\cr}%
  }%
}

\begin{document}

 \begin{center}
    \Large\textbf{1.36}
 \end{center}

Compute the value of $2^{(p-1)/2}$ (mod p) for every prime 3 $\leq$ p $<$ 20. Make a conjecture as to the possible values of $2^{(p-1)/2}$ (mod p) when p is prime and prove that your conjecture is correct.

\begin{center}
  $2^{(p-1)/2} = r$ (mod p)
\end{center}

\begin{table}[h!]
  \begin{center}
    \caption{3 $\leq$ $p$ $<$ 20}
    \begin{tabular}{l|l}
      $p$ & $r$  \\
      \hline
      3 & 2 \\
      5 & 4 \\
      7 & 1 \\
      11 & 10 \\
      13 & 12 \\
      17 & 1 \\
      19 & 18 \\
    \end{tabular}
  \end{center}
\end{table}

Conjecture: $r \equiv \pm1$ (mod p) where $p$ is prime.

\bigskip

Proof: Let $r = 2^{(p-1)/2}$. Then $r^2 = 2^{(p-1)}$ by simplification. Since 2 is prime, $p \ndivides 2$. Then by Fermat's Little Theorem, $r^2 = 2^{(p-1)} \equiv 1$ (mod p). Thus $x^2 \equiv 1$ (mod p). Now $r \equiv \pm1$ (mod p) by $1^2 = 1$ and $-1^2 = 1$.

\end{document}
